\documentclass[a4paper, 11pt]{article}
\usepackage[T1]{fontenc} 
\usepackage[utf8]{inputenc}
\usepackage[english]{babel}%Veröffentlichungssprache
\usepackage{csquotes}
\usepackage{graphicx}
\usepackage{ragged2e}
\usepackage[format=plain,justification=RaggedRight,singlelinecheck=false,font={small},labelsep=space]{caption}
\usepackage[dvipsnames]{xcolor}	
\usepackage[a4paper]{geometry}
	\geometry{left=3.5cm,right=2.5cm,top=2.4cm,bottom=2cm}%Seitenränder
	\usepackage[onehalfspacing]{setspace}%Zeilenabstand
	\renewcommand{\\}{\vspace*{0.5\baselineskip} \newline}
\renewcommand*\MakeUppercase[1]{#1}	
\usepackage{fancyhdr}
	\pagestyle{fancy}
	\renewcommand{\headrulewidth}{0pt}
	\renewcommand{\footrulewidth}{0pt}
	\fancyhead[R]{\footnotesize{\thepage}}
	\fancyhead[L]{\footnotesize{\leftmark}}
	\fancyfoot{}
\usepackage[colorlinks,
pdfpagelabels,
pdfstartview = FitH,
bookmarksopen = true,
bookmarksnumbered = true,
linkcolor = black,
urlcolor = black,
plainpages = false,
hypertexnames = false,
citecolor = black] {hyperref}
\usepackage[parfill]{parskip}
\usepackage{listings}
\usepackage[
	backend=biber,
	style=apa
]{biblatex}
\addbibresource{refs.bib}

\renewcommand{\familydefault}{\sfdefault}

\begin{document}

\title{Code signatures for plagiarism detection}
\author{Dennis Goßler \and Dennis Wäckerle}
\maketitle

\section*{Abstract}
\newpage
\tableofcontents
\newpage

\section{Introduction}

\section{How Does Plagiarism Detection Work}

\section{Use Case and Software Experiment}

\section{Criteria for Evaluation}

\section{Evaluation of the different Tools}

\subsection{MOSS}

\subsection{JPlag}

\subsection{Plaggie}

Hallo das ist ein Test \autocite[Hallo 4]{JPlagP}.
Hallo das ist ein Test \autocite[Hallo 4]{JPlagP}.
Hallo das ist ein Test \autocite[Hallo 4]{JPlagP}.

Hallo das ist ein Test \autocite[Hallo 4]{JPlagP}.

\subsection{AC2}

\section{Conclusion}

\newpage

\printbibliography[
	heading=bibintoc,
	title={References}
]

\appendix

\end{document}